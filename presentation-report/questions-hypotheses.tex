\section{Research questions and hypotheses}

\begin{frame}[allowframebreaks]{Research questions and hypotheses}
    \begin{itemize}
        \item What is the relation between the amount of requests that RoboPizza receives and the waiting time for customers?
        \begin{itemize}
                \item $H_1$: The waiting time increases with the amount of requests.
                \item $H_0$: The waiting time does \textbf{not} increase with the amount of requests.
        \end{itemize}

        \item Do robots drive more (non-idle time) when there are more requests in the system?
        \begin{itemize}
                \item $H_1$: Robots drive more when there are more requests in the system.
                \item $H_0$: Robots do \textbf{not} driving more when there are more requests in the system.
        \end{itemize}

        \framebreak

        \item Does increasing the amount of robots decrease customer waiting time when there are many requests?
        \begin{itemize}
                \item $H_1$: Increasing the amount of robots decreases customer waiting time when there are many requests.
                \item $H_0$: Increasing the amount of robots does \textbf{not} decrease customer waiting time when there are many requests.
        \end{itemize}

        \item How do waiting times change as the amount of road works changes (dynamism)?
        \begin{itemize}
            \item $H_1$: Waiting times increase as the amount of road works increase.
            \item $H_0$: Waiting times do \textbf{not} increase as the amount of road works increase.
        \end{itemize}
    \end{itemize}
\end{frame}
